\documentclass[12pt,a4paper]{article}
\usepackage{ctex}
\usepackage{amsmath, amssymb, amsthm}
\usepackage{geometry}
\geometry{margin=1in}
\setlength{\parindent}{0pt}
\setlength{\headheight}{15pt}
\usepackage{fancyhdr}
\pagestyle{fancy}
\fancyhf{}
\fancyhead[L]{PB22000150 刘行}
\fancyhead[C]{NPDE 第 07 次作业}
\fancyhead[R]{第~\thepage~页}

\begin{document}
    {\LARGE \textbf{HW07}}\vspace{0.5em}

    \section*{4.3.1}
                取能量
        \begin{equation*}
            E\left(t\right)=\left\lVert u\left(\cdot, t\right) \right\rVert_{L^{2}}^{2}=\int u\left(x, t\right)^{\ast}u\left(x,t\right)\,\mathrm{d}x.
        \end{equation*}
        则有
        \begin{equation*}
            \dot E\left(t\right)=2 \, \Re\left\langle u_{t},u\right\rangle = 2 \,\Re\int u^{\ast}\left(A u_{x}+Bu\right) \, \mathrm{d}x.
        \end{equation*}

        \textbf{(1) 关于 $A u_{x}$ 项:}
        \begin{equation*}
            \int u^{\ast}A u_{x} \, \mathrm{d}x.
        \end{equation*}
        当 $A$ 为常矩阵时, 若 $A=A^{\ast}$, 则
        \begin{equation*}
            u^{\ast}A u_{x}=\tfrac12 \partial_{x}\left(u^{\ast}A u\right),
        \end{equation*}
        于是
        \begin{equation*}
            2\,\Re\int u^{\ast}A u_{x} \, \mathrm{d}x = \Re\int \partial_{x}\left(u^{\ast}A u\right) \, \mathrm{d}x = \Re\,[u^{\ast}A u]_{\text{边界}}.
        \end{equation*}
        在周期边界, 无穷远衰减或零通量边界条件下, 该边界项为零.
        因此要保证能量守恒, 必须有
        \begin{equation*}
            \boxed{A=A^{\ast}.}
        \end{equation*}

        \textbf{(2) 关于 $Bu$ 项:}
        \begin{equation*}
            2 \, \Re\int u^{\ast}Bu \, \mathrm{d}x = \int u^{\ast}\left(B+B^{\ast}\right)u \, \mathrm{d}x.
        \end{equation*}
        要使该项恒为零, 必须且只需
        \begin{equation*}
            \boxed{B^{\ast} = -B.}
        \end{equation*}

        \textbf{(3) 结论:}

        在适当的边界条件下 (如周期, 衰减或零通量), 系统
        \begin{equation*}
            u_{t} = A u_{x} + B
        \end{equation*}
        能量守恒当且仅当
        \begin{equation*}
            \boxed{A\ \text{自伴},\qquad B\ \text{反自伴}.}
        \end{equation*}
        等价地, 生成算子
        \begin{equation*}
            L=A\partial_{x}+B
        \end{equation*}
        在 $L^{2}$ 空间中反自伴.


    \section*{4.4.1}
        设能量
        \begin{equation*}
            E\left(t\right) = \left\lVert u\left(\cdot,t\right) \right\rVert^{2} = \left\langle u,u\right\rangle.
        \end{equation*}
        对时间求导得
        \begin{equation*}
            \frac{1}{2}\frac{\mathrm{d}}{\mathrm{d}t}\left\lVert u\right\rVert^{2} = \Re\left\langle u_{t},u\right\rangle = \Re\left\langle A u_{xx},u\right\rangle.
        \end{equation*}
        对右端分部积分 (假设边界项为零):
        \begin{equation*}
            \Re\left\langle A u_{xx},u\right\rangle = -\Re\left\langle A u_{x},u_{x}\right\rangle.
        \end{equation*}
        若 $A$ 是常矩阵且自伴正定, 即 $A=A^{\ast}$, 且存在 $\alpha>0$ 使得
        \begin{equation*}
            \left\langle Av,v\right\rangle\ge \alpha \left\lVert v \right\rVert^{2},
        \end{equation*}
        代入得,
        \begin{equation*}
            \frac{\mathrm{d}}{\mathrm{d}t}\left\lVert u\right\rVert^{2}+2\alpha\left\lVert u_{x} \right\rVert^{2}\le 0.
        \end{equation*}
        积分 $t\in\left[0,T\right]$ 得
        \begin{equation*}
            \left\lVert u\left(\cdot,t\right)\right\rVert^{2} + 2\alpha\int_{0}^{t}\left\lVert u_{x}\left(\cdot,\xi\right) \right\rVert^{2} \, \mathrm{d}\xi \le \left\lVert u\left(\cdot,0\right) \right\rVert^{2}.
        \end{equation*}
        因此式(4.4.9)成立, 可取
        \begin{equation*}
            \boxed{\delta = 2\alpha = 2\lambda_{\min}\left(A\right),\qquad K = 1.}
        \end{equation*}

    \section*{4.4.2}
        考虑系统
        \begin{equation*}
            u_{t}=A u_{xx}+B u_{x}+C u,
        \end{equation*}
        其中 $B=B^{\ast}$, $C^{\ast}=-C$. 同样有
        \begin{equation*}
            \frac{1}{2}\frac{\mathrm{d}}{\mathrm{d}t}\left\lVert u \right\rVert^{2} = \Re\left\langle A u_{xx},u\right\rangle + \Re\left\langle B u_{x},u\right\rangle + \Re\left\langle C u, u\right\rangle.
        \end{equation*}
        对各项分析:

        \begin{itemize}
            \item 第一项: 与4.4.1相同,
            \begin{equation*}
                \Re\left\langle Au_{xx}, u\right\rangle = -\Re\left\langle Au_{x}, u_{x}\right\rangle \le -\alpha\left\lVert u_{x} \right\rVert^{2}.
            \end{equation*}
            \item 第二项: 由于$B$为常矩阵且$B=B^{\ast}$,
            \begin{equation*}
                \Re\left\langle Bu_{x}, u\right\rangle = \frac{1}{2} \Re\int \partial_{x}\left(u^{\ast} Bu\right) \, \mathrm{d}x=0,
            \end{equation*}
            边界项为零.
            \item 第三项: 因$C^{\ast}=-C$,
            \begin{equation*}
                u^{\ast}Cu\text{为纯虚数}\Rightarrow \Re\left\langle Cu, u\right\rangle=0.
            \end{equation*}
        \end{itemize}

        因此
        \begin{equation*}
            \frac{\mathrm{d}}{\mathrm{d}t}\left\lVert u\right\rVert^{2}+2\alpha\left\lVert u_{x} \right\rVert^{2}\le 0,
        \end{equation*}
        与4.4.1完全相同. 积分得
        \begin{equation*}
            \left\lVert u\left(\cdot,t\right)\right\rVert^{2}+2\alpha\int_{0}^{t}\left\lVert u_{x}\left(\cdot,\xi\right) \right\rVert^{2} \, \mathrm{d}\xi \le \left\lVert u\left(\cdot,0\right) \right\rVert^{2}.
        \end{equation*}
        故式(4.4.9)依然成立, 并可取相同的常数
        \begin{equation*}
            \boxed{\delta=2\lambda_{\min}\left(A\right),\qquad K=1.}
        \end{equation*}

    \section*{4.5.1}
        对系统 $u_{t} = A \, u_{x}$ 进行空间傅里叶变换:
        \begin{equation*}
            \widehat{u}_{t} = i\omega A \, \widehat{u}.
        \end{equation*}
        因此其符号矩阵为 $\widehat{P}\left(i\omega\right) = i\omega A$, 特征值为
        \begin{equation*}
            \lambda = i\omega \mu,
        \end{equation*}
        其中 $\mu$ 是矩阵 $A$ 的特征值. 若 $\mu = a + ib$, 则
        \begin{equation*}
            \Re\lambda = \Re\left(i\omega\mu\right) = -\omega \, \Im\mu = -\omega b.
        \end{equation*}
        若 $b \neq 0$, 即 $A$ 存在非实特征值, 则可以改变 $\omega$ 的符号, 使 $\Re\lambda$ 取得任意大的正值, 从而不存在有限的 $\alpha$ 使得 $\Re\lambda \le \alpha$ 对所有 $\omega$ 成立.

        若 $b=0$, 即 $A$ 的特征值全为实数, 则 $\Re\lambda = 0$, 此时 Petrovsky 条件在 $\alpha = 0$ 时已经成立.

        因此不可能出现 ``当 $\alpha>0$ 时成立而当 $\alpha=0$ 不成立'' 的情况. 结论是:

        \begin{equation*}
            \boxed{\text{若 }A\text{ 的特征值全实, 则条件在 }\alpha=0\text{ 时成立; 否则对任意 }\alpha\text{ 都不成立.}}
        \end{equation*}

    \section*{4.5.2}
        给定系统
        \begin{equation*}
            u_{t} =
            \begin{bmatrix}
                1 & 10\\
                0 & 2
            \end{bmatrix}
            u_{x},
        \end{equation*}
        有
        \begin{equation*}
            \widehat{P}\left(i\omega\right) = i\omega A, \qquad
            A =
            \begin{bmatrix}
                1 & 10\\
                0 & 2
            \end{bmatrix}.
        \end{equation*}

        根据定理 4.5.7, 我们需要找到一个正定厄米矩阵 $\widehat{H}\left(\omega\right)$ (为方便起见记为 $H$, 与 $\omega$ 无关)使得
        \begin{equation*}
            H A = A^{\ast} H.
        \end{equation*}
        此时
        \begin{equation*}
            H\,\widehat{P}\left(i\omega\right) + \widehat{P}\left(i\omega\right)^{\ast} H = i\omega\left(HA - A^{\ast} H\right) = 0 \le 2\alpha H,
        \end{equation*}
        因此可以取 $\alpha = 0$.

        设
        \begin{equation*}
            H =
            \begin{bmatrix}
                h_{11} & h_{12}\\
                h_{12} & h_{22}
            \end{bmatrix}, \qquad H = H^{\ast}.
        \end{equation*}
        代入 $HA = A^{\top} H$, 得到方程组:
        \begin{equation*}
            \begin{cases}
                10h_{11} + 2h_{12} = h_{12},\\
                h_{12} = 10h_{11} + 2h_{12}.
            \end{cases}
        \end{equation*}
        由此解得 $h_{12} = -10h_{11}$, 而 $h_{22}$ 可任意取.
        为了保证 $H$ 正定, 只需取 $h_{22} > 100h_{11}$.
        例如, 取 $h_{11}=1$, $h_{22}=101$, 得
        \begin{equation*}
            H =
            \begin{bmatrix}
                1 & -10\\
                -10 & 101
            \end{bmatrix}.
        \end{equation*}

        显然 $H$ 是厄米正定矩阵, 且满足 $H A = A^{\top} H$.
        于是有
        \begin{equation*}
            H\,\widehat{P}\left(i\omega\right) + \widehat{P}\left(i\omega\right)^{\ast} H = 0 \le 2\alpha H \quad \left(\alpha = 0\right),
        \end{equation*}
        并且由于 $H$ 正定, 存在常数 $K > 0$ 使
        \begin{equation*}
            K^{-1}I \le H \le K I.
        \end{equation*}

        因此, 矩阵
        \begin{equation*}
            \boxed{
                \widehat{H}\left(\omega\right) =
                \begin{bmatrix}
                    1 & -10\\
                    -10 & 101
                \end{bmatrix}
            }
        \end{equation*}
        满足定理 4.5.7 的条件 (4.5.14) 和 (4.5.15),
        且可取 $\alpha = 0$.

    \section*{补充作业}
        设系统经过傅里叶变换后为
        \begin{equation*}
            \partial_{t} \widehat{u}\left(\omega, t\right) = \widehat{P}\left(i\omega\right) \, \widehat{u}\left(\omega, t\right),
        \end{equation*}

        定义能量函数
        \begin{equation*}
            E\left(\omega, t\right) := \left\lVert \widehat{u}\left(\omega, t\right) \right\rVert_{\mathbb{C}^{m}}^{2} = \widehat{u}\left(\omega, t\right)^{\ast}\widehat{u}\left(\omega, t\right).
        \end{equation*}
        对时间求导, 有
        \begin{equation*}
            \frac{\mathrm{d}}{\mathrm{d}t}E\left(\omega,t\right) = \widehat{u}^{\ast}\widehat{P}\left(i\omega\right)\widehat{u} + \left(\widehat{P}\left(i\omega\right)\widehat{u}\right)^{\ast}\widehat{u} = \widehat{u}^{\ast}\left(\widehat{P}\left(i\omega\right)+\widehat{P}\left(i\omega\right)^{\ast}\right)\widehat{u}.
        \end{equation*}
        由假设条件,
        \begin{equation*}
            \frac{\mathrm{d}}{\mathrm{d}t}E\left(\omega, t\right) \le 2\alpha\,E\left(\omega, t\right).
        \end{equation*}
        由 Gronwall 不等式可得
        \begin{equation*}
            E\left(\omega, t\right) \le e^{2\alpha t}E\left(\omega, 0\right) = e^{2\alpha t}\left\lVert \widehat{u}_{0}\left(\omega\right) \right\rVert^{2}.
        \end{equation*}

        对 $\omega$ 积分, 并由 Parseval 等式,
        \begin{equation*}
            \left\lVert u\left(\cdot, t\right) \right\rVert_{L^{2}}^{2} = \int_{\mathbb{R}} E\left(\omega, t\right) \, \mathrm{d}\omega \le e^{2\alpha t}\int_{\mathbb{R}} \left\lVert \widehat{u}_{0}\left(\omega\right) \right\rVert^{2} \, \mathrm{d}\omega = e^{2\alpha t}\left\lVert u_{0} \right\rVert_{L^{2}}^{2}.
        \end{equation*}
        因此,
        \begin{equation*}
            \boxed{\left\lVert u\left(\cdot, t\right)\right\rVert_{L^{2}} \le e^{\alpha t}\left\lVert u_{0} \right\rVert_{L^{2}}.}
        \end{equation*}

        该不等式表明:
        \begin{itemize}
            \item 解随时间的增长至多呈指数形式;
            \item 解对初值连续依赖;
            \item 解唯一.
        \end{itemize}

        由于常系数线性系统的存在性可由傅里叶表示
        \begin{equation*}
            \widehat{u}\left(\omega, t\right) = e^{t\widehat{P}\left(i\omega\right)}\widehat{u}_{0}\left(\omega\right)
        \end{equation*}
        直接得到, 因此初值问题 (1) 是 \textbf{well-posed}.

        事实上, 这正是定理 4.5.7 的特例, 取 $\widehat{H}\left(\omega\right) \equiv I$, $K=1$,
        此时条件
        \begin{equation*}
            \widehat{H}\widehat{P}\left(i\omega\right) + \widehat{P}\left(i\omega\right)^{\ast}\widehat{H} \le 2\alpha \widehat{H}
        \end{equation*}
        即为题设所给的不等式.
\end{document}
