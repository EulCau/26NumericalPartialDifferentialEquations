\documentclass[12pt]{article}
\usepackage{ctex}
\usepackage{amsmath,amssymb,amsthm}
\usepackage[a4paper,margin=2.5cm]{geometry}

\title{NPDE 作业 10}
\author{刘行\quad PB22000150}
\date{}

\begin{document}
\maketitle

\section*{题目1 \;(HW 4.4.11)}

证明三维 Peaceman--Rachford 格式
\begin{equation*}
\left(1-\frac{r_x}{3}\delta_x^2\right)u_{jkl}^{\,n+\frac13}
=
\left(1+\frac{r_y}{3}\delta_y^2+\frac{r_z}{3}\delta_z^2\right)u_{jkl}^{\,n},
\end{equation*}
\begin{equation*}
\left(1-\frac{r_y}{3}\delta_y^2\right)u_{jkl}^{\,n+\frac23}
=
\left(1+\frac{r_x}{3}\delta_x^2+\frac{r_z}{3}\delta_z^2\right)u_{jkl}^{\,n+\frac13},
\end{equation*}
\begin{equation*}
\left(1-\frac{r_z}{3}\delta_z^2\right)u_{jkl}^{\,n+1}
=
\left(1+\frac{r_x}{3}\delta_x^2+\frac{r_y}{3}\delta_y^2\right)u_{jkl}^{\,n+\frac23},
\end{equation*}
对三维热方程
\[
u_t=a(u_{xx}+u_{yy}+u_{zz})
\]
是条件稳定的,并且截断误差阶为
\[
O(\Delta t)+O(\Delta x^2)+O(\Delta y^2)+O(\Delta z^2).
\]

\subsection*{(1) 截断误差与相容性}

记空间步长分别为 $\Delta x,\Delta y,\Delta z$,时间步长为 $\Delta t$,
\[
r_x=\frac{a\Delta t}{\Delta x^2},\quad
r_y=\frac{a\Delta t}{\Delta y^2},\quad
r_z=\frac{a\Delta t}{\Delta z^2},
\]
三点中心差分算子
\[
\delta_x^2 u_{jkl}^n=\frac{u_{j+1,kl}^n-2u_{jkl}^n+u_{j-1,kl}^n}{\Delta x^2},
\]
$\delta_y^2,\delta_z^2$ 类似。

把恰解 $u(x,y,z,t)$ 代入第一步格式,在网格点
$(x_j,y_k,z_l,t_n)$ 处展开 Taylor 公式:
\begin{equation*}
u^{n+\frac13}_{jkl}
=
u_{jkl}(t_n+\tfrac13\Delta t)
=
u_{jkl}^n+\tfrac13\Delta t\,u_t
+O(\Delta t^2),
\end{equation*}
\begin{equation*}
\delta_x^2 u^{n+\frac13}_{jkl}
=
u_{xx}(t_n+\tfrac13\Delta t)+O(\Delta x^2),
\,
\delta_y^2 u^{n}_{jkl}
=
u_{yy}(t_n)+O(\Delta y^2),
\,
\delta_z^2 u^{n}_{jkl}
=
u_{zz}(t_n)+O(\Delta z^2).
\end{equation*}
代入第一式左、右两边:
\begin{align*}
\text{LHS}
&=\left(1-\frac{r_x}{3}\delta_x^2\right)u^{n+\frac13}_{jkl}  \\
&=
u_{jkl}^n+\tfrac13\Delta t\,u_t
- \frac{a\Delta t}{3}\,u_{xx}
+O(\Delta t^2)+O(\Delta x^2),
\\[0.3em]
\text{RHS}
&=\left(1+\frac{r_y}{3}\delta_y^2+\frac{r_z}{3}\delta_z^2\right)u_{jkl}^{n}  \\
&=
u_{jkl}^n
+\frac{a\Delta t}{3}(u_{yy}+u_{zz})
+O(\Delta y^2)+O(\Delta z^2).
\end{align*}
利用 PDE 本身
\[
u_t=a(u_{xx}+u_{yy}+u_{zz}),
\]
在 $t_n$ 处有
\[
\tfrac13\Delta t\,u_t
-\frac{a\Delta t}{3}u_{xx}
-\frac{a\Delta t}{3}(u_{yy}+u_{zz})=0.
\]
于是第一步方程的局部截断误差为
\[
\tau^{(1)}_{jkl}=O(\Delta t^2)+O(\Delta x^2)+O(\Delta y^2)+O(\Delta z^2).
\]
第二、三步完全类似,只是时间层分别为 $t_n+\tfrac13\Delta t$ 与
$t_n+\tfrac23\Delta t$,对这两个时刻同样有 PDE 成立,因此三步的局部截断误差均为
\[
O(\Delta t^2)+O(\Delta x^2)+O(\Delta y^2)+O(\Delta z^2).
\]

由分步格式组合得到整体格式的一步时间推进,
其主导误差项仍为 $O(\Delta t)$,空间为二阶。
因此该三维 Peaceman--Rachford 格式是
\[
\text{一阶时间、二阶空间的相容格式: }
O(\Delta t)+O(\Delta x^2)+O(\Delta y^2)+O(\Delta z^2).
\]

\subsection*{(2) von~Neumann 稳定性分析}

做冻结系数与 Fourier 模分析。对每个空间方向取周期边界,
令
\[
u_{jkl}^n = g^n 
\exp\big(i(j\theta_x+k\theta_y+l\theta_z)\big),
\]
记
\[
\lambda_x=-4\sin^2\frac{\theta_x}{2},\quad
\lambda_y=-4\sin^2\frac{\theta_y}{2},\quad
\lambda_z=-4\sin^2\frac{\theta_z}{2},
\]
则
\[
\delta_x^2 u_{jkl}^n=\lambda_x u_{jkl}^n,\quad
\delta_y^2 u_{jkl}^n=\lambda_y u_{jkl}^n,\quad
\delta_z^2 u_{jkl}^n=\lambda_z u_{jkl}^n,
\]
其中 $\lambda_x,\lambda_y,\lambda_z\le0$。

记
\[
a_x=\frac{r_x}{3}\lambda_x\le0,\quad
a_y=\frac{r_y}{3}\lambda_y\le0,\quad
a_z=\frac{r_z}{3}\lambda_z\le0.
\]
三步格式在该 Fourier 模上的标量形式为
\begin{align*}
(1-a_x)u^{n+\frac13} &= (1+a_y+a_z)u^n,\\
(1-a_y)u^{n+\frac23} &= (1+a_x+a_z)u^{n+\frac13},\\
(1-a_z)u^{n+1}      &= (1+a_x+a_y)u^{n+\frac23}.
\end{align*}
消去中间层,得到增益因子
\begin{equation*}
G:=\frac{u^{n+1}}{u^n}
=
\frac{(1+a_y+a_z)(1+a_x+a_z)(1+a_x+a_y)}
     {(1-a_x)(1-a_y)(1-a_z)}.
\end{equation*}
注意 $a_x,a_y,a_z\le0$,于是分母各因子均大于等于 $1$,
分子中每个因子形如 $1+$(非正数)。
当 $\Delta t\to0$ 时,$r_x,r_y,r_z\to0$,
从而 $a_x,a_y,a_z\to0$,展开得
\begin{equation*}
G
=1+O(r_x^2+r_y^2+r_z^2).
\end{equation*}
因此在 $\Delta t$ 足够小时,
$|G|\le1$,并且 $G$ 关于 $(r_x,r_y,r_z)$ 是连续函数,
故存在常数 $C>0$,当
\[
r_x+r_y+r_z\le C
\quad
\Big(\text{即 }\Delta t\le C_0(\Delta x^2+\Delta y^2+\Delta z^2)\Big)
\]
时,对所有 Fourier 模都有 $|G|\le1$。
因此该格式在满足适当 CFL 条件($\Delta t$ 充分小)时稳定,
即为 \textbf{条件稳定}。

\bigskip

\section*{题目2 \;(HW 2.4.1)}

证明当 $|R|\le1$ 时,差分格式
\begin{equation*}
u_k^{n+1}
=\frac12\big(u_{k+1}^n+u_{k-1}^n\big)
-\frac{R}{2}\,\delta_0 u_k^n
\end{equation*}
关于 $\|\cdot\|_\infty$ 范是稳定的。这里
\[
\delta_0 u_k^n = u_{k+1}^n-u_{k-1}^n.
\]

\subsection*{证明}

先把格式写成显式凸组合形式。
\begin{align*}
u_k^{n+1}
&=\frac12(u_{k+1}^n+u_{k-1}^n)
-\frac{R}{2}(u_{k+1}^n-u_{k-1}^n) \\
&=\left(\frac12-\frac{R}{2}\right)u_{k+1}^n
 +\left(\frac12+\frac{R}{2}\right)u_{k-1}^n.
\end{align*}
记
\[
\alpha=\frac{1-R}{2},\qquad
\beta =\frac{1+R}{2},
\]
则
\[
u_k^{n+1}=\alpha\,u_{k+1}^n+\beta\,u_{k-1}^n.
\]

当 $|R|\le1$ 时,有
\[
\alpha\ge0,\quad \beta\ge0,\quad \alpha+\beta=1.
\]
因此
\begin{align*}
|u_k^{n+1}|
&=|\alpha\,u_{k+1}^n+\beta\,u_{k-1}^n|
\le \alpha|u_{k+1}^n|+\beta|u_{k-1}^n|
\le (\alpha+\beta)\max_j|u_j^n|\\
&=\max_j|u_j^n|.
\end{align*}
对所有 $k$ 取最大值得
\[
\|u^{n+1}\|_\infty
\le \|u^n\|_\infty.
\]
从而
\[
\|u^{n}\|_\infty \le \|u^0\|_\infty,\qquad n=1,2,\dots
\]
该格式在 $|R|\le1$ 时关于 sup 范是非增的,因此是稳定的。

\bigskip

\section*{题目3 \;(补充题 1)}

\textbf{构造 $u_t+u_x+u_y=0$ 的 ADI 格式,并分析其精度与稳定性。}

考虑二维平流方程
\[
u_t+u_x+u_y=0,
\]
假设平流速度为常数 $1$,取网格
\[
x_j=j\Delta x,\quad y_k=k\Delta y,\quad t_n=n\Delta t.
\]
为保证稳定性,在空间上采用迎风差分,
因为速度为正,所以
\[
D_x^-u_{j,k}^n
=\frac{u_{j,k}^n-u_{j-1,k}^n}{\Delta x},\qquad
D_y^-u_{j,k}^n
=\frac{u_{j,k}^n-u_{j,k-1}^n}{\Delta y}.
\]

\subsection*{(1) ADI 分裂格式}

采用分两步的 ADI 思想:
\begin{itemize}
  \item 第一步:在 $x$ 方向隐式、$y$ 方向显式,
  \begin{equation*}
  \frac{u_{j,k}^{\,n+\frac12}-u_{j,k}^{\,n}}{\Delta t/2}
  + D_x^-u_{j,k}^{\,n+\frac12}
  + D_y^-u_{j,k}^{\,n}=0.
  \end{equation*}
  \item 第二步:在 $y$ 方向隐式、$x$ 方向显式,
  \begin{equation*}
  \frac{u_{j,k}^{\,n+1}-u_{j,k}^{\,n+\frac12}}{\Delta t/2}
  + D_x^-u_{j,k}^{\,n+\frac12}
  + D_y^-u_{j,k}^{\,n+1}=0.
  \end{equation*}
\end{itemize}
这是一个典型的超抛物方程 ADI 迎风格式。
由两步组合可知,该格式时间离散为一阶精度,
空间上 $D_x^-,D_y^-$ 为一阶迎风差分,
因此总体精度为
\[
O(\Delta t)+O(\Delta x)+O(\Delta y).
\]

\subsection*{(2) von~Neumann 稳定性分析}

同样做 Fourier 模分析。
令
\[
u_{j,k}^n = G^n
\exp\big(i(j\theta_x+k\theta_y)\big),
\]
则
\[
D_x^- u_{j,k}^n = a_x u_{j,k}^n,\qquad
D_y^- u_{j,k}^n = a_y u_{j,k}^n,
\]
其中
\begin{align*}
a_x &= \frac{1-e^{-i\theta_x}}{\Delta x},
\\
a_y &= \frac{1-e^{-i\theta_y}}{\Delta y}.
\end{align*}
注意
\[
\Re(a_x)=\frac{1-\cos\theta_x}{\Delta x}\ge0,\quad
\Re(a_y)=\frac{1-\cos\theta_y}{\Delta y}\ge0.
\]

设在第 $n$ 步的 Fourier 系数为 $g^n$。
第一步在该模上的标量形式为
\[
\frac{g^{n+\frac12}-g^n}{\Delta t/2}
+ a_x g^{n+\frac12}+a_y g^n =0,
\]
整理得
\[
\bigg(1+\frac{\Delta t}{2}a_x\bigg)g^{n+\frac12}
=
\bigg(1-\frac{\Delta t}{2}a_y\bigg)g^n.
\]
第二步为
\[
\frac{g^{n+1}-g^{n+\frac12}}{\Delta t/2}
+ a_x g^{n+\frac12}+a_y g^{n+1}=0,
\]
即
\[
\bigg(1+\frac{\Delta t}{2}a_y\bigg)g^{n+1}
=
\bigg(1-\frac{\Delta t}{2}a_x\bigg)g^{n+\frac12}.
\]
两式相乘得到一步增益因子
\begin{equation*}
G
:=\frac{g^{n+1}}{g^n}
=
\frac{\big(1-\frac{\Delta t}{2}a_x\big)
      \big(1-\frac{\Delta t}{2}a_y\big)}
     {\big(1+\frac{\Delta t}{2}a_x\big)
      \big(1+\frac{\Delta t}{2}a_y\big)}.
\end{equation*}

考虑复数 $\alpha$ 满足 $\Re(\alpha)\ge0$,
则
\[
\left|\frac{1-\alpha}{1+\alpha}\right|\le1.
\]
这里 $a_x,a_y$ 的实部均非负,故
\[
\left|\frac{1-\frac{\Delta t}{2}a_x}{1+\frac{\Delta t}{2}a_x}\right|\le1,
\qquad
\left|\frac{1-\frac{\Delta t}{2}a_y}{1+\frac{\Delta t}{2}a_y}\right|\le1,
\]
于是
\[
|G|\le1.
\]
这对任意 $\Delta t>0$ 成立,
说明本 ADI 迎风格式在 von~Neumann 意义下是
\textbf{无条件稳定} 的。

\bigskip

\section*{题目4 \;(补充题 2)}

已知非线性抛物方程
\[
u_t=a(u)u_{xx},
\]
其数值格式为(冻结系数法)
\begin{equation*}
v_j^{n+1}
= v_j^{n}
+\frac12\mu a\big(u_j^{n+\frac12}\big)
\Big(\delta_x^2 v_j^{n+1}+\delta_x^2 v_j^n\Big),
\end{equation*}
其中
\[
\mu=\frac{\Delta t}{\Delta x^2},\qquad
v_j^{n+\frac12}=\frac32v_j^n-\frac12v_j^{n-1},
\]
$\delta_x^2$ 为三点中心差分。
利用冻结系数法分析该格式的 $L_2$ 模稳定性和最大模稳定性。

\subsection*{(1) 冻结系数与线性化}

在冻结系数法中,把 $a(u_j^{n+\frac12})$ 看成在某个固定值处的常数,
记为 $\tilde a>0$,则格式化为线性:
\begin{equation*}
\frac{v_j^{n+1}-v_j^{n}}{\Delta t}
=
\frac{\tilde a}{2}
\Big(\delta_x^2 v_j^{n+1}+\delta_x^2 v_j^n\Big).
\end{equation*}
这正是常系数热方程
\[
u_t=\tilde a u_{xx}
\]
的 Crank--Nicolson 格式。因而可以直接套用 CN 格式的稳定性分析。

\subsection*{(2) $L_2$ 模稳定性}

记向量 $v^n=(\dots,v_j^n,\dots)^\top$,
采用标准离散内积
\[
(v,w)=\sum_j v_j w_j\,\Delta x,\qquad
\|v\|_2^2=(v,v).
\]
离散 Laplace 算子
\[
(\delta_x^2 v)_j=\frac{v_{j+1}-2v_j+v_{j-1}}{\Delta x^2}
\]
是对称负定的,即
\[
(\delta_x^2 v,v)\le0.
\]

把格式写成向量形式:
\[
\frac{v^{n+1}-v^{n}}{\Delta t}
=
\frac{\tilde a}{2}\Big(L v^{n+1}+L v^n\Big),
\]
其中 $L$ 为离散 Laplace 矩阵。
与 $v^{n+1}+v^n$ 取内积:
\begin{align*}
\left(\frac{v^{n+1}-v^{n}}{\Delta t},\,v^{n+1}+v^n\right)
&=
\frac{\tilde a}{2}\big(L (v^{n+1}+v^n),\,v^{n+1}+v^n\big).
\end{align*}
左边化简为
\[
\frac{1}{\Delta t}\Big(\|v^{n+1}\|_2^2-\|v^n\|_2^2\Big),
\]
右边利用 $L$ 的对称性得
\[
\frac{\tilde a}{2}\big(L w,w\big)\le0,\qquad
w=v^{n+1}+v^n.
\]
于是
\[
\frac{1}{\Delta t}
\Big(\|v^{n+1}\|_2^2-\|v^n\|_2^2\Big)
\le0,
\]
即
\[
\|v^{n+1}\|_2\le\|v^n\|_2.
\]
因而在冻结系数意义下,该格式对任意 $\Delta t>0$ 都是
\textbf{$L_2$ 无条件稳定} 的。

\subsection*{(3) 最大模稳定性}

对一维周期网格做 Fourier 分析。
设
\[
v_j^n = g^n e^{ij\theta},\qquad \theta\in[-\pi,\pi].
\]
则
\[
\delta_x^2 v_j^n
=
\lambda(\theta)v_j^n,\qquad 
\lambda(\theta)=\frac{e^{i\theta}-2+e^{-i\theta}}{\Delta x^2}
=\frac{-4\sin^2(\theta/2)}{\Delta x^2}\le0.
\]
代入线性化格式:
\[
\frac{g^{n+1}-g^n}{\Delta t}
=\frac{\tilde a}{2}\lambda(\theta)
\big(g^{n+1}+g^n\big).
\]
整理得
\[
\bigg(1-\frac{\tilde a\Delta t}{2}\lambda(\theta)\bigg)g^{n+1}
=
\bigg(1+\frac{\tilde a\Delta t}{2}\lambda(\theta)\bigg)g^{n}.
\]
记
\[
\rho(\theta)
:=\frac{g^{n+1}}{g^n}
=
\frac{1+\frac{\tilde a\Delta t}{2}\lambda(\theta)}
     {1-\frac{\tilde a\Delta t}{2}\lambda(\theta)}.
\]
由于 $\lambda(\theta)\le0$,设
\[
\alpha=-\frac{\tilde a\Delta t}{2}\lambda(\theta)\ge0,
\]
则
\[
\rho(\theta)=\frac{1-\alpha}{1+\alpha},
\qquad
|\rho(\theta)|=\frac{|1-\alpha|}{1+\alpha}\le1.
\]
可见每个 Fourier 模的振幅不增。
在周期情形下,数值解可以写成这些模的线性组合,
因此系数的放大因子均不超过 $1$,
从而 sup 范也不会增长:
\[
\|v^{n+1}\|_\infty\le\|v^n\|_\infty.
\]
因此在冻结系数意义下,该格式对最大模也是稳定的。

\end{document}
