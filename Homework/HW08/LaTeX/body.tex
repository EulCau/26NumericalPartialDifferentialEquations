\documentclass[12pt]{article}

\usepackage[UTF8]{ctex}
\usepackage{amsmath,amssymb,amsthm}
\usepackage{geometry}
\geometry{a4paper,scale=0.8}

\title{偏微分方程数值解作业}
\author{PB22000150~~刘行}
\date{\today}

\begin{document}
    \maketitle

    \section*{3.1.2}
        考虑偏微分方程
        \begin{equation*}
        v_t + a v_x = \nu v_{xx},
        \end{equation*}
        令
        \begin{equation*}
        R = \frac{a\,\Delta t}{\Delta x},\qquad
        r = \frac{\nu\,\Delta t}{\Delta x^2},
        \end{equation*}
        并采用中心差分算子
        \begin{equation*}
        \delta_0 v_k^n = v_{k+1}^n - v_{k-1}^n,\qquad
        \delta^2 v_k^n = v_{k+1}^n - 2v_k^n + v_{k-1}^n .
        \end{equation*}

        在 von~Neumann 意义下,令
        \begin{equation*}
        v_k^n = G^n e^{i k\theta},\qquad \theta = \frac{2\pi m}{M},
        \end{equation*}
        则
        \begin{equation*}
        \delta_0 v_k^n = 2i\sin\theta\, v_k^n,\qquad
        \delta^2 v_k^n = -4\sin^2\!\left(\frac{\theta}{2}\right) v_k^n .
        \end{equation*}

        \subsection*{(1) 格式一的稳定性}
            格式为
            \begin{equation*}
            v_k^{n+1} + \frac{R}{2}\delta_0 v_k^{n+1} - r\,\delta^2 v_k^{n+1} = v_k^n .
            \end{equation*}
            代入 $v_k^{n+1} = G v_k^n$,并使用上面的差分作用结果,得到
            \begin{equation*}
            \left[1 + \frac{R}{2}\cdot 2i\sin\theta - r\bigl(-4\sin^2(\tfrac{\theta}{2})\bigr)\right] G v_k^n
            = v_k^n .
            \end{equation*}
            约去 $v_k^n\neq 0$,得
            \begin{equation*}
            \bigl(1 + i R\sin\theta + 4r\sin^2(\tfrac{\theta}{2})\bigr) G = 1,
            \end{equation*}
            即放大因子
            \begin{equation*}
            G(\theta) =
            \frac{1}{1 + 4r\sin^2(\tfrac{\theta}{2}) + i R\sin\theta } .
            \end{equation*}
            于是
            \begin{equation*}
            |G(\theta)|^2
            = \frac{1}{\bigl(1 + 4r\sin^2(\tfrac{\theta}{2})\bigr)^2 + R^2\sin^2\theta } .
            \end{equation*}
            注意到 $r\ge 0$,$\sin^2(\tfrac{\theta}{2})\ge 0$,故
            \begin{equation*}
            \bigl(1 + 4r\sin^2(\tfrac{\theta}{2})\bigr)^2 \ge 1,
            \end{equation*}
            再加上非负项 $R^2\sin^2\theta$ 得
            \begin{equation*}
            \bigl(1 + 4r\sin^2(\tfrac{\theta}{2})\bigr)^2 + R^2\sin^2\theta \ge 1.
            \end{equation*}
            因此
            \begin{equation*}
            |G(\theta)|^2 \le 1,\qquad \forall\,\theta,\ \forall\,r\ge 0,\ R\in\mathbb{R},
            \end{equation*}
            说明该格式无条件稳定。

        \subsection*{(2) 格式二的稳定性}
            格式为
            \begin{equation*}
            v_k^{n+1} + \frac{R}{4}\delta_0 v_k^{n+1}
            - \frac{r}{2}\delta^2 v_k^{n+1}
            = v_k^n - \frac{R}{4}\delta_0 v_k^n + \frac{r}{2}\delta^2 v_k^n .
            \end{equation*}
            同样代入
            \begin{equation*}
            v_k^{n+1} = G v_k^n,\qquad
            \delta_0 v_k^n = 2i\sin\theta\, v_k^n,\qquad
            \delta^2 v_k^n = -4\sin^2(\tfrac{\theta}{2})v_k^n,
            \end{equation*}
            得到左端为
            \begin{equation*}
            \left[1 + \frac{R}{4}\cdot 2i\sin\theta
                - \frac{r}{2}\bigl(-4\sin^2(\tfrac{\theta}{2})\bigr)\right]G v_k^n
            = \bigl(1 + i\frac{R}{2}\sin\theta + 2r\sin^2(\tfrac{\theta}{2})\bigr) G v_k^n ,
            \end{equation*}
            右端为
            \begin{equation*}
            \left[1 - \frac{R}{4}\cdot 2i\sin\theta
                + \frac{r}{2}\bigl(-4\sin^2(\tfrac{\theta}{2})\bigr)\right] v_k^n
            = \bigl(1 - i\frac{R}{2}\sin\theta - 2r\sin^2(\tfrac{\theta}{2})\bigr) v_k^n .
            \end{equation*}
            约去 $v_k^n$,得到
            \begin{equation*}
            \bigl(1 + i\frac{R}{2}\sin\theta + 2r\sin^2(\tfrac{\theta}{2})\bigr) G
            = 1 - i\frac{R}{2}\sin\theta - 2r\sin^2(\tfrac{\theta}{2}).
            \end{equation*}
            记
            \begin{equation*}
            s = \sin\!\left(\frac{\theta}{2}\right),\qquad
            B = \frac{R}{2}\sin\theta,\qquad
            A = 2rs^2,
            \end{equation*}
            则
            \begin{equation*}
            G(\theta) =
            \frac{1 - A - iB}{1 + A + iB}.
            \end{equation*}
            于是
            \begin{equation*}
            |G(\theta)|^2
            = \frac{(1-A)^2 + B^2}{(1+A)^2 + B^2}.
            \end{equation*}
            因为 $A = 2rs^2\ge 0$,故
            \begin{equation*}
            (1-A)^2 \le (1+A)^2,
            \end{equation*}
            于是
            \begin{equation*}
            (1-A)^2 + B^2 \le (1+A)^2 + B^2,
            \end{equation*}
            从而
            \begin{equation*}
            |G(\theta)|^2 \le 1,\qquad \forall\,\theta,\ \forall\,r\ge 0,\ R\in\mathbb{R}.
            \end{equation*}

            因此,题目中的两个差分格式均为无条件稳定。

    \section*{补 1}
        考虑线性平流方程
        \begin{equation*}
        u_t + a u_x = 0 ,\qquad a \in \mathbb{R}.
        \end{equation*}
        Lax 格式写为
        \begin{equation*}
        v_j^{n+1} = \frac12 \bigl( v_{j+1}^n + v_{j-1}^n \bigr)
                - \frac12 r \delta_0 v_j^n,
        \end{equation*}
        其中
        \begin{equation*}
        r = \frac{a\Delta t}{\Delta x},\qquad
        \delta_0 v_j^n = v_{j+1}^n - v_{j-1}^n .
        \end{equation*}

        \subsection*{(1) 将格式写成凸组合形式}
            把格式中的 $\delta_0 v_j^n$ 展开, 有
            \begin{equation*}
            v_j^{n+1}
            = \frac12 (v_{j+1}^n + v_{j-1}^n)
                - \frac12 r (v_{j+1}^n - v_{j-1}^n)
            = \alpha v_{j+1}^n + \beta v_{j-1}^n ,
            \end{equation*}
            其中
            \begin{equation*}
            \alpha = \frac12 (1-r),\qquad
            \beta  = \frac12 (1+r).
            \end{equation*}
            容易验证
            \begin{equation*}
            \alpha + \beta = 1 .
            \end{equation*}
            当
            \begin{equation*}
            |r| = \left|\frac{a\Delta t}{\Delta x}\right| \le 1
            \end{equation*}
            时, 有
            \begin{equation*}
            1-r \ge 0,\qquad 1+r \ge 0,
            \end{equation*}
            于是
            \begin{equation*}
            \alpha \ge 0,\qquad \beta \ge 0.
            \end{equation*}
            因此在 CFL 条件 $|r|\le 1$ 下,
            \begin{equation*}
            v_j^{n+1} = \alpha v_{j+1}^n + \beta v_{j-1}^n
            \end{equation*}
            是两个格点值的凸组合.

        \subsection*{(2) 离散最大模原理与 $L_\infty$ 稳定性}
            定义离散无穷范数
            \begin{equation*}
            \left\lVert v^n \right\rVert_\infty = \max_{j} |v_j^n|.
            \end{equation*}
            有
            \begin{equation*}
            |v_j^{n+1}| = |\alpha v_{j+1}^n + \beta v_{j-1}^n|
            \le \alpha |v_{j+1}^n| + \beta |v_{j-1}^n|.
            \end{equation*}
            再利用 $\alpha,\beta\ge 0$ 以及
            \begin{equation*}
            |v_{j+1}^n| \le \left\lVert v^n \right\rVert_\infty ,\qquad
            |v_{j-1}^n| \le \left\lVert v^n \right\rVert_\infty ,
            \end{equation*}
            得到
            \begin{equation*}
            |v_j^{n+1}|
            \le (\alpha + \beta)\,\left\lVert v^n \right\rVert_\infty
            = \left\lVert v^n \right\rVert_\infty .
            \end{equation*}
            对所有 $j$ 取最大值得
            \begin{equation*}
            \left\lVert v^{n+1}\right\rVert _\infty \le  \left\lVert v^n \right\rVert_\infty .
            \end{equation*}
            用数学归纳法可得, 对任意 $n\ge 0$,
            \begin{equation*}
            \left\lVert v^n\right\rVert _\infty \le  \left\lVert v^0 \right\rVert_\infty .
            \end{equation*}
            因此, 当 $|r|\le 1$ 时, Lax 格式关于 $L_\infty$ 范数是稳定的.

    \section*{补 2}
        考虑热方程
        \begin{equation*}
        u_t = u_{xx}.
        \end{equation*}
        其 FTCS 格式为
        \begin{equation*}
        v_j^{n+1}
        = v_j^n + \mu (v_{j+1}^n - 2v_j^n + v_{j-1}^n),
        \end{equation*}
        其中
        \begin{equation*}
        \mu = \frac{\Delta t}{\Delta x^2}.
        \end{equation*}
        定义离散 $L_2$ 范数
        \begin{equation*}
        \left\lVert v^n \right\rVert_{2,\Delta x}^2
        = \sum_j |v_j^n|^2\,\Delta x .
        \end{equation*}

        \subsection*{(1) Fourier 模分析}
            设格点函数在无穷区间或周期区间上, 可写成 Fourier 模的线性组合. 只需对单一波数 $k$ 的模
            \begin{equation*}
            v_j^n = \hat{v}^n e^{i k x_j},\qquad x_j = j\Delta x
            \end{equation*}
            进行分析. 代入格式, 有
            \begin{equation*}
            \hat{v}^{n+1} e^{i k x_j}
            = \hat{v}^n e^{i k x_j}
            + \mu \hat{v}^n
            \left( e^{i k x_{j+1}} -2 e^{i k x_j} + e^{i k x_{j-1}} \right).
            \end{equation*}
            两边约去 $e^{i k x_j}$, 得到放大因子 $G(k)$:
            \begin{equation*}
            \hat{v}^{n+1} = G(k)\,\hat{v}^n,
            \qquad
            G(k) = 1 + \mu\left( e^{i k\Delta x} -2 + e^{-i k\Delta x} \right).
            \end{equation*}
            利用
            \begin{equation*}
            e^{i\theta} + e^{-i\theta} = 2\cos\theta ,
            \end{equation*}
            有
            \begin{equation*}
            G(k)
            = 1 + \mu \bigl( 2\cos(k\Delta x) - 2 \bigr)
            = 1 + 2\mu\bigl( \cos(k\Delta x)-1 \bigr).
            \end{equation*}
            进一步利用
            \begin{equation*}
            \cos(k\Delta x)-1
            = -2\sin^2\!\left(\frac{k\Delta x}{2}\right),
            \end{equation*}
            得到
            \begin{equation*}
            G(k)
            = 1 - 4\mu \sin^2\!\left(\frac{k\Delta x}{2}\right).
            \end{equation*}
            可见 $G(k)$ 为实数, 且满足
            \begin{equation*}
            G(k)\le 1.
            \end{equation*}
            要保证稳定性, 需要对所有波数 $k$ 有
            \begin{equation*}
            |G(k)| \le 1.
            \end{equation*}
            因为
            \begin{equation*}
            0 \le \sin^2\!\left(\frac{k\Delta x}{2}\right) \le 1,
            \end{equation*}
            所以
            \begin{equation*}
            G_{\min} = 1 - 4\mu .
            \end{equation*}
            因此
            \begin{equation*}
            |G(k)|\le1 \Longleftrightarrow
            -1 \le 1 - 4\mu \le 1,
            \end{equation*}
            化简得到
            \begin{equation*}
            0 \le \mu \le \frac12.
            \end{equation*}

        \subsection*{(2) $L_{2,\Delta x}$ 范数的稳定性}
            若 $0\le\mu\le \tfrac12$, 则对任意波数 $k$ 有 $|G(k)|\le1$.
            离散 Fourier 变换是一个保持 $L_2$ 范数的正交变换, 因此
            \begin{equation*}
            \left\lVert v^{n+1} \right\rVert_{2,\Delta x}^2
            = \sum_k |\hat{v}^{n+1}(k)|^2
            = \sum_k |G(k)\hat{v}^n(k)|^2
            \le \sum_k |\hat{v}^n(k)|^2
            = \left\lVert v^n \right\rVert_{2,\Delta x}^2.
            \end{equation*}
            于是对任意 $n$,
            \begin{equation*}
            \left\lVert v^n\right\rVert _{2,\Delta x} \le  \left\lVert v^0 \right\rVert_{2,\Delta x},
            \end{equation*}
            即在条件
            \begin{equation*}
            0\le\mu=\frac{\Delta t}{\Delta x^2} \le \frac12
            \end{equation*}
            下, 热方程的 FTCS 格式关于 $L_{2,\Delta x}$ 是稳定的.

    \section*{补3}
        考虑线性常系数方程
        \begin{equation*}
        u_t + u_x - \nu_2 u_{xx} + \mu_3 u_{xxx} = 0,
        \qquad \nu_2,\mu_3 \text{ 为常数}.
        \end{equation*}

        \subsection*{(1) Fourier 模分析}
            令
            \begin{equation*}
            u(x,t) = \exp\bigl( i(kx-\omega t)\bigr),
            \end{equation*}
            其中 $k$ 为波数, $\omega$ 为频率. 则
            \begin{equation*}
            u_t = -i\omega u,\quad
            u_x = ik u,\quad
            u_{xx} = (ik)^2 u = -k^2 u,\quad
            u_{xxx} = (ik)^3 u = -ik^3 u .
            \end{equation*}
            代回 PDE:
            \begin{equation*}
            (-i\omega)u + (ik)u - \nu_2(-k^2u) + \mu_3(-ik^3u)=0.
            \end{equation*}
            约去 $u\neq0$ 得
            \begin{equation*}
            -i\omega + ik + \nu_2 k^2 - i\mu_3 k^3 = 0.
            \end{equation*}
            把上述式子写成
            \begin{equation*}
            -i\omega = -ik - \nu_2 k^2 + i\mu_3 k^3 ,
            \end{equation*}
            两边乘以 $i$, 得到色散关系
            \begin{equation*}
            \omega = k - \mu_3 k^3 + i(-\nu_2 k^2).
            \end{equation*}
            记
            \begin{equation*}
            \omega = \omega_R + i\omega_I,
            \end{equation*}
            则
            \begin{equation*}
            \omega_R = k - \mu_3 k^3,\qquad
            \omega_I = -\nu_2 k^2.
            \end{equation*}

        \subsection*{(2) 耗散性分析}
            解的一般形式为
            \begin{equation*}
            u(x,t) = \exp\bigl(-i\omega t\bigr)\exp(ikx)
                = \exp\bigl(-i\omega_R t\bigr)
                    \exp\bigl(\omega_I t\bigr)\exp(ikx).
            \end{equation*}
            其中振幅因子为 $\exp(\omega_I t)$. 若对所有 $k$ 有 $\omega_I\le0$,
            则振幅随时间衰减, 方程是耗散的.

            这里
            \begin{equation*}
            \omega_I = -\nu_2 k^2 \le 0 \quad (\nu_2 \ge 0),
            \end{equation*}
            且当 $k\neq0$ 时严格小于 0. 因此当 $\nu_2>0$ 时, 该方程对非零波数表现为指数衰减, 是耗散的.

        \subsection*{(3) 色散性分析}
            相速度为
            \begin{equation*}
            c_p(k) = \frac{\omega_R}{k}
                = 1 - \mu_3 k^2.
            \end{equation*}
            可见 $c_p$ 随波数 $k$ 变化, 只要 $\mu_3\neq0$, 不同波数的信号传播速度不同, 这就是典型的色散现象. 因此:
            \begin{itemize}
            \item $\nu_2 > 0$ 决定耗散;
            \item $\mu_3 \neq 0$ 决定色散.
            \end{itemize}

    \section*{补 4}
        \subsection*{(1) PDE $u_t = u_{xx}$ 的耗散性与色散性}
            将方程写成
            \begin{equation*}
            u_t - u_{xx} = 0.
            \end{equation*}
            同样取 Fourier 模
            \begin{equation*}
            u(x,t) = \exp\bigl(i(kx-\omega t)\bigr),
            \end{equation*}
            则
            \begin{equation*}
            u_t = -i\omega u,\qquad
            u_{xx} = -k^2 u.
            \end{equation*}
            代入 PDE:
            \begin{equation*}
            -i\omega u - (-k^2 u) = 0
            \quad\Longrightarrow\quad
            -i\omega + k^2 = 0.
            \end{equation*}
            于是
            \begin{equation*}
            \omega = -i k^2.
            \end{equation*}
            故
            \begin{equation*}
            \omega_R = 0,\qquad
            \omega_I = -k^2 \le 0.
            \end{equation*}
            振幅因子为
            \begin{equation*}
            \exp(\omega_I t) = \exp(-k^2 t),
            \end{equation*}
            对所有 $k\neq0$ 均随时间指数衰减, 说明热方程是强耗散的.

            另一方面, 相速度
            \begin{equation*}
            c_p = \frac{\omega_R}{k} = 0,
            \end{equation*}
            与波数无关, 且没有传播, 只是纯衰减, 因此该 PDE 不具有色散性 (无相速度的波数依赖性).

        \subsection*{(2) FTCS 格式的耗散性与色散性 (方法一: 模分析)}
            FTCS 格式表达式仍为
            \begin{equation*}
            v_j^{n+1}
            = v_j^n + \mu (v_{j+1}^n - 2v_j^n + v_{j-1}^n),
            \qquad
            \mu = \frac{\Delta t}{\Delta x^2}.
            \end{equation*}
            对 Fourier 模
            \begin{equation*}
            v_j^n = \hat{v}^n e^{ikx_j}
            \end{equation*}
            进行分析, 之前已得到放大因子
            \begin{equation*}
            G(k) = 1 - 4\mu \sin^2\!\left(\frac{k\Delta x}{2}\right).
            \end{equation*}
            记
            \begin{equation*}
            G(k) = |G(k)| e^{-i\theta(k)},
            \end{equation*}
            则数值解在一步时间上的变化为
            \begin{equation*}
            \hat{v}^{n+1} = G(k)\,\hat{v}^n
            = |G(k)| e^{-i\theta(k)} \hat{v}^n .
            \end{equation*}
            其中
            \begin{equation*}
            |G(k)| = \bigl|1-4\mu\sin^2(\tfrac{k\Delta x}{2})\bigr|
            \end{equation*}
            控制振幅变化, $\theta(k)$ 控制相位变化.

            \paragraph{耗散性} 当 $0<\mu<\tfrac12$ 时, 对非零波数有
            \begin{equation*}
            0 < 1 - 4\mu \le G(k) < 1,
            \end{equation*}
            特别地, 当 $k\neq0$ 且 $\sin^2(\tfrac{k\Delta x}{2})>0$ 时, 有
            \begin{equation*}
            0 < G(k) < 1,
            \end{equation*}
            故
            \begin{equation*}
            |G(k)|<1.
            \end{equation*}
            这意味着每一步中所有非零波数的振幅都被衰减, 表现出数值耗散性, 而且高频 ($\sin^2$ 较大) 衰减更快.

            \paragraph{色散性} 由于 $G(k)$ 为实数且非负, 所以
            \begin{equation*}
            \theta(k) = 0,
            \end{equation*}
            一步时间更新中没有额外的相位改动, 因此从严格意义上讲, 这个 FTCS 格式对热方程并不引入数值色散 (相速度始终为 0), 只表现为纯耗散.

        \subsection*{(3) FTCS 格式的耗散性与色散性 (方法二: 修正方程)}
            另一种分析方法是通过求修正方程 (又称等效方程). 将数值格式展开到一定阶数, 寻找一个连续 PDE, 使其解在截断误差范围内逼近数值解.

            对格式
            \begin{equation*}
            v_j^{n+1}
            = v_j^n + \mu (v_{j+1}^n - 2v_j^n + v_{j-1}^n)
            \end{equation*}
            假设 $v_j^n$ 近似于某个光滑函数 $u(x,t)$ 在 $(x_j,t_n)$ 处的值:
            \begin{equation*}
            v_j^n \approx u(x_j,t_n).
            \end{equation*}
            对时间方向作 Taylor 展开:
            \begin{equation*}
            u(x_j,t_{n+1})
            = u(x_j,t_n) + \Delta t\,u_t
            + \frac{\Delta t^2}{2} u_{tt}
            + O(\Delta t^3).
            \end{equation*}
            空间方向同理:
            \begin{equation*}
            u(x_{j\pm1},t_n)
            = u(x_j,t_n) \pm \Delta x\,u_x
            + \frac{\Delta x^2}{2} u_{xx}
            \pm \frac{\Delta x^3}{6} u_{xxx}
            + \frac{\Delta x^4}{24} u_{xxxx}
            + O(\Delta x^5).
            \end{equation*}
            代入格式
            \begin{equation*}
            u(x_j,t_{n+1})
            = u(x_j,t_n) + \mu
            \bigl( u(x_{j+1},t_n) -2u(x_j,t_n) + u(x_{j-1},t_n) \bigr),
            \end{equation*}
            左边用时间展开, 右边用空间展开, 得到
            \begin{equation*}
            u + \Delta t\,u_t + \frac{\Delta t^2}{2} u_{tt} + O(\Delta t^3)
            = u + \mu\bigl( \Delta x^2 u_{xx} + \frac{\Delta x^4}{12}u_{xxxx} + O(\Delta x^5) \bigr).
            \end{equation*}
            约去 $u$, 得到
            \begin{equation*}
            \Delta t\,u_t + \frac{\Delta t^2}{2} u_{tt}
            = \mu\Delta x^2 u_{xx} + \mu\frac{\Delta x^4}{12} u_{xxxx}
            + O(\Delta t^3,\Delta x^5).
            \end{equation*}
            记 $\mu = \dfrac{\Delta t}{\Delta x^2}$, 化简得
            \begin{equation*}
            u_t + \frac{\Delta t}{2} u_{tt}
            = u_{xx} + \frac{\Delta x^2}{12} u_{xxxx}
            + O(\Delta t^2,\Delta x^4).
            \end{equation*}
            现用原 PDE $u_t = u_{xx}$ 代入 $u_{tt}$ 的主阶:
            \begin{equation*}
            u_{tt} = (u_{xx})_t \approx (u_t)_{xx} = (u_{xx})_{xx} = u_{xxxx}.
            \end{equation*}
            于是修正方程近似为
            \begin{equation*}
            u_t = u_{xx}
            + \left(\frac{\Delta x^2}{12}-\frac{\Delta t}{2}\right) u_{xxxx}
            + \cdots .
            \end{equation*}

            \paragraph{从修正方程看耗散性与色散性}

            对 Fourier 模 $u = e^{i(kx-\omega t)}$,
            \begin{equation*}
            u_{xx} = -k^2 u,\qquad
            u_{xxxx} = k^4 u.
            \end{equation*}
            代入修正方程主阶:
            \begin{equation*}
            -i\omega u
            = -k^2 u
            + \left(\frac{\Delta x^2}{12}-\frac{\Delta t}{2}\right) k^4 u.
            \end{equation*}
            约去 $u$,
            \begin{equation*}
            -i\omega
            = -k^2 + \left(\frac{\Delta x^2}{12}-\frac{\Delta t}{2}\right)k^4.
            \end{equation*}
            从中可以看出:
            \begin{itemize}
            \item 右侧全为实数, 因此 $\omega$ 为纯虚数, 说明修正方程仍然没有相位项, 不产生色散;
            \item 若
                \begin{equation*}
                \frac{\Delta x^2}{12}-\frac{\Delta t}{2}<0
                \quad\Longleftrightarrow\quad
                \Delta t > \frac{\Delta x^2}{6},
                \end{equation*}
                则高阶项的符号与主耗散项相同 (都是 $-k^2$ 型), 增强了高频耗散;
            \item 反之若 $\Delta t$ 很小, 则高阶项可能减弱部分波数的耗散, 但在稳定条件 $\mu\le\tfrac12$ 下, 总体仍表现为耗散.
            \end{itemize}
            从修正方程角度再次验证: FTCS 格式主要引入的是高阶耗散项而非色散项, 与模分析的结论一致.
\end{document}
