\documentclass[12pt,a4paper]{article}
\usepackage{amsmath,amssymb,geometry}
\geometry{margin=2.5cm}
\setlength{\parindent}{2em}

\begin{document}

\noindent \textbf{HW03}

\vspace{1em}

\noindent \textbf{2.2.1} \quad 
Prove that the solution of the leap-frog scheme converges to the solution of the differential equation, if $\lambda \le 1-\delta$, $\delta > 0$.

\medskip

Consider the linear advection equation
\[
u_t + a u_x = 0,
\]
discretized by the leap-frog scheme:
\[
\frac{u_j^{n+1} - u_j^{n-1}}{2\Delta t} 
+ a\,\frac{u_{j+1}^n - u_{j-1}^n}{2\Delta x} = 0.
\]
Let $\lambda = a\Delta t / \Delta x$. The update formula is
\[
u_j^{n+1} = u_j^{n-1} - \lambda (u_{j+1}^n - u_{j-1}^n).
\]

\textbf{(1) Consistency.}
By Taylor expansion,
\[
\frac{u(t+\Delta t)-u(t-\Delta t)}{2\Delta t}
= u_t + \tfrac{1}{6}u_{ttt}\Delta t^2 + O(\Delta t^4),
\]
and similarly for the space derivative. Hence the local truncation error is
\[
\tau = O(\Delta t^2 + \Delta x^2),
\]
which tends to zero as $\Delta t, \Delta x \to 0$. Thus, the scheme is second-order accurate and consistent with the PDE.

\textbf{(2) Stability.}
Let $u_j^n = \rho^n e^{i k x_j}$, where $\theta = k \Delta x$.  
Substitute into the scheme to obtain
\[
\rho^2 + 2i\lambda \sin\theta \,\rho - 1 = 0.
\]
Solving for $\rho$ gives
\[
\rho = -i\lambda \sin\theta \pm \sqrt{1 - \lambda^2 \sin^2\theta}.
\]
If $|\lambda \sin\theta| \le 1$, then
\[
|\rho|^2 = (\lambda \sin\theta)^2 + (1 - \lambda^2 \sin^2\theta) = 1,
\]
so each Fourier mode is neutrally stable.  
If $\lambda > 1$, some modes satisfy $|\lambda \sin\theta| > 1$, leading to exponential growth.

Therefore, for $\lambda \le 1-\delta$ with $\delta>0$, all modes remain bounded uniformly in time.  
By the Lax equivalence theorem, since the scheme is consistent and stable, it converges to the true solution of the differential equation.

\bigskip
\noindent \textbf{2.2.2} \quad 
Derive the explicit form of the leap-frog approximation for $\lambda=1$. Is the scheme suitable for computation?

\medskip

When $\lambda=1$, the amplification equation becomes
\[
\rho^2 + 2i\sin\theta\,\rho - 1 = 0,
\]
giving
\[
\rho = -i\sin\theta \pm \sqrt{1 - \sin^2\theta}
      = -i\sin\theta \pm |\cos\theta|.
\]
Equivalently, the two roots can be expressed as
\[
\rho_1 = e^{-i\theta}, \qquad 
\rho_2 = -e^{i\theta}.
\]
Hence the general Fourier solution is
\[
\rho^n = A e^{-in\theta} + B(-1)^n e^{in\theta}.
\]
The first term corresponds to the physical mode propagating correctly,  
while the second term $(-1)^n e^{in\theta}$ is a \emph{computational mode}, 
oscillating in sign at each time step.

Although $|\rho|=1$ (neutral stability), this mode does not decay and can be excited by round-off or boundary errors, producing spurious oscillations.  
Thus, when $\lambda=1$, the scheme is theoretically neutrally stable but practically unsuitable for computation unless additional filtering or special startup procedures are applied.

\end{document}
