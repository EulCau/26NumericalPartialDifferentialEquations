\documentclass[12pt]{article}
\usepackage{ctex}
\usepackage{amsmath, amsfonts, amssymb}
\usepackage{geometry}
\geometry{a4paper, margin=1in}

\title{偏微分方程数值解作业09}
\author{PB22000150~刘行}
\date{\today}

\begin{document}
\maketitle

\section*{HW 5.8.4}

题目给出的差分格式为
\begin{equation*}
u^{n+1}_{j,k}
= u^n_{j,k}
- \frac{R_x}{2}\delta_x^0 u^n_{j,k}
+ \frac{R_x^2}{2}\delta_x^2 u^n_{j,k}
- \frac{R_y}{2}\delta_y^0 u^n_{j,k}
+ \frac{R_y^2}{2}\delta_y^2 u^n_{j,k}.
\end{equation*}
其中
\begin{equation*}
\delta_x^0 u^n_{j,k} = u^n_{j+1,k} - u^n_{j-1,k}, \qquad
\delta_x^2 u^n_{j,k} = u^n_{j+1,k} - 2u^n_{j,k} + u^n_{j-1,k},
\end{equation*}
$y$ 方向定义完全类似。

该格式试图逼近二维线性平流方程
\begin{equation*}
u_t + a u_x + b u_y = 0,
\qquad
R_x = \frac{a\Delta t}{\Delta x},\;
R_y = \frac{b\Delta t}{\Delta y}.
\end{equation*}

我们对格式左侧做 Taylor 展开:
\begin{align*}
u^{n+1}_{j,k}
&= u(x_j,y_k,t_n+\Delta t) \\
&= u + \Delta t u_t + \frac{\Delta t^2}{2}u_{tt}
+ \frac{\Delta t^3}{6}u_{ttt} + O(\Delta t^4).
\end{align*}

再展开右侧每个差分算子。以 $x$ 方向为例,
\begin{align*}
\delta_x^0 u^n_{j,k}
&= u(x+\Delta x) - u(x-\Delta x) \\
&= 2\Delta x u_x + \frac{2\Delta x^3}{3}u_{xxx} + O(\Delta x^5),
\\
\delta_x^2 u^n_{j,k}
&= u(x+\Delta x) - 2u(x) + u(x-\Delta x)
= \Delta x^2 u_{xx} + \frac{\Delta x^4}{12}u_{xxxx} + O(\Delta x^6).
\end{align*}

代入格式右端得:
\begin{align*}
u^n_{j,k}
&- R_x\Delta x\, u_x - \frac{R_x\Delta x^3}{3}u_{xxx}
+ \frac{R_x^2}{2}\Delta x^2 u_{xx}
- R_y\Delta y \,u_y - \frac{R_y\Delta y^3}{3}u_{yyy}
+ \frac{R_y^2}{2}\Delta y^2 u_{yy}
+ O(\Delta x^4 + \Delta y^4).
\end{align*}

由于 $R_x=a\Delta t/\Delta x$,化简得
\begin{align*}
u^n_{j,k}
&- a\Delta t u_x - \frac{a\Delta t\,\Delta x^2}{3}u_{xxx}
+ \frac{a^2\Delta t^2}{2} u_{xx}
- b\Delta t u_y - \frac{b\Delta t\,\Delta y^2}{3}u_{yyy}
+ \frac{b^2\Delta t^2}{2} u_{yy}.
\end{align*}

现在比较两边 Taylor 展开。左侧二阶项为
\begin{equation*}
\frac{\Delta t^2}{2}u_{tt}.
\end{equation*}

而 PDE 给出
\begin{equation*}
u_{tt} = (a u_x + b u_y)_t = a u_{xt} + b u_{yt}
= a^2 u_{xx} + 2 ab u_{xy} + b^2 u_{yy}.
\end{equation*}

于是左侧二阶项应为
\begin{equation*}
\frac{\Delta t^2}{2}
(a^2 u_{xx} + 2 a b u_{xy} + b^2 u_{yy}).
\end{equation*}

而右侧二阶项为
\begin{equation*}
\frac{a^2\Delta t^2}{2}u_{xx}
+ \frac{b^2\Delta t^2}{2}u_{yy},
\end{equation*}
缺失关键项
\begin{equation*}
a b \Delta t^2 u_{xy}.
\end{equation*}

因此二阶项不匹配,从而局部截断误差包含一阶项:
\begin{equation*}
\tau = O(\Delta t),
\end{equation*}
故该格式 \textbf{在时间上不是二阶精度}。

\section*{HW 5.8.7}

我们希望确定点 $(j\Delta x,k\Delta y,(n+1)\Delta t)$ 的数值解依赖哪些网格点。

\subsection*{(1) 格式一}
\begin{equation*}
u^{n+1}_{j,k}
= a_2 u^n_{j,k-1} + a_3 u^n_{j,k} + a_4 u^n_{j,k+1}.
\end{equation*}

显然,它仅用到上一层 $n$ 层的 $(j,k-1)$、$(j,k)$、$(j,k+1)$。

继续向前递推可得依赖域为所有满足
\begin{equation*}
k-(n-m) \le k' \le k+(n-m)
\end{equation*}
的点 $(j,k')$,其中 $m$ 为层数。

因此数值依赖域为以 $(j,k)$ 为中心、$k$ 方向逐层扩大的条带区域(横向三点格式形成 1D 锥形域)。

\subsection*{(2) 格式二}
\begin{equation*}
u^{n+1}_{j,k}
= a_1 u^n_{j-1,k} + a_3 u^n_{j,k} + a_4 u^n_{j+1,k} + a_5 u^n_{j,k+1}.
\end{equation*}

该格式依赖四个点:
\begin{equation*}
(j-1,k),\quad (j,k),\quad (j+1,k),\quad (j,k+1).
\end{equation*}

逐层回溯可知依赖域呈 ``L'' 形扩展,既包含 $x$ 方向的扩散,也包含 $y$ 方向的偏移。

\section*{补充题}

考虑热方程
\begin{equation*}
u_t = \alpha u_{xx}.
\end{equation*}

Crank--Nicolson 格式写为
\begin{equation*}
\frac{u^{n+1}_j - u^n_j}{\Delta t}
= \alpha \frac{u^{n+1}_{j+1}-2u^{n+1}_j+u^{n+1}_{j-1}}{2\Delta x^2}
+ \alpha \frac{u^n_{j+1}-2u^n_j+u^n_{j-1}}{2\Delta x^2}.
\end{equation*}

\subsection*{1. 截断误差}

对 $u^{n+1}_j$ 做 Taylor 展开:
\begin{equation*}
u^{n+1}_j
= u + \Delta t u_t + \frac{\Delta t^2}{2}u_{tt}
+ O(\Delta t^3),
\end{equation*}
空间二阶差分展开为
\begin{equation*}
\frac{u_{j+1}-2u_j+u_{j-1}}{\Delta x^2}
= u_{xx} + \frac{\Delta x^2}{12}u_{xxxx} + O(\Delta x^4).
\end{equation*}

代入后可得局部截断误差
\begin{equation*}
\tau = O(\Delta t^2 + \Delta x^2).
\end{equation*}

因此 CN 格式 \textbf{时间二阶、空间二阶}。

\subsection*{2. 相容性}

将 PDE $u_t = \alpha u_{xx}$ 代入差分格式,可发现主项完全相消,仅余高阶项,因此该格式相容于热方程。

由于截断误差 $\tau \to 0$ 当 $(\Delta t, \Delta x)\to 0$,故满足相容性定义。

\subsection*{3. 稳定性(Fourier 模分析)}

令
\begin{equation*}
u^n_j = \hat{u}^n e^{i k j\Delta x}.
\end{equation*}
代入可得放大因子
\begin{equation*}
G = \frac{1 - r\sin^2(k\Delta x/2)}{1 + r\sin^2(k\Delta x/2)},
\qquad r = \alpha\frac{\Delta t}{\Delta x^2}.
\end{equation*}

显然有
\begin{equation*}
|G|\le 1.
\end{equation*}

因此 CN 格式为 **无条件稳定**。

\bigskip

综上,Crank--Nicolson 格式具有:
\begin{itemize}
  \item 二阶时间精度;
  \item 二阶空间精度;
  \item 无条件稳定;
  \item 与热方程相容。
\end{itemize}

\end{document}
