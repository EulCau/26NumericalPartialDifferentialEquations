\documentclass[12pt,a4paper]{article}
\usepackage{ctex}
\usepackage{amsmath, amssymb, amsthm}
\usepackage{geometry}
\geometry{margin=1in}
\setlength{\parindent}{0pt}

\begin{document}
    \textbf{HW06}\vspace{0.5em}

    \section*{4.1.1}
        考虑傅里叶模态$u(x,t)=e^{i\omega x}\widehat u(\omega,t)$,代入方程$u_t = u_{xx} + 100u$得:
        \begin{equation*}
            \frac{d}{dt}\widehat u(\omega,t)=(100-\omega^2)\widehat u(\omega,t)
        \end{equation*}
        精确解为$\widehat u(\omega,t)=e^{(100-\omega^2)t}\widehat f(\omega)$。

        分两种情形分析舍入误差:

        情形A:初始数据存在舍入误差$\delta$,即实际初始振幅为$\widehat f(\omega)+\delta$。最终时刻$t$的数值解为:
        \begin{equation*}
            \widehat u_{\text{num}}(\omega,t)=e^{(100-\omega^2)t}(\widehat f(\omega)+\delta)
        \end{equation*}
        相对于精确解$\widehat u(\omega,t)=e^{(100-\omega^2)t}\widehat f(\omega)$的相对误差为:
        \begin{equation*}
            \frac{|\widehat u_{\text{num}}(\omega,t)-\widehat u(\omega,t)|}{|\widehat u(\omega,t)|}
            =\frac{|\delta|}{|\widehat f(\omega)|}
        \end{equation*}
        要保证相对误差不超过$1\%$,需满足:
        \begin{equation*}
            \frac{|\delta|}{|\widehat f(\omega)|}\le 0.01
        \end{equation*}

        情形B:时间推进过程中产生局部舍入误差$\varepsilon(s)$。在时间$s$引入的误差到最终时刻$t$会被放大$e^{(100-\omega^2)(t-s)}$倍。
        最坏情况下($\omega=0$),最大放大因子为$e^{100\cdot 2}=e^{200}$。
        为保证最终相对误差不超过$1\%$,保守条件要求每次局部绝对误差$\varepsilon$满足:
        \begin{equation*}
            \varepsilon \le 0.01\cdot e^{-200}|u(\cdot,2)|
        \end{equation*}
        其中$e^{-200}\approx 1.38\times10^{-87}$。

    \section*{4.2.1}
        考虑偏微分方程$\frac{\partial u}{\partial t}=\sum_{j=0}^4 a_j\frac{\partial^j u}{\partial x^j}$。
        设解的形式为$u(x,t)=e^{i\omega x}\widehat u(\omega,t)$,代入方程得:
        \begin{equation*}
            \frac{d}{dt}\widehat u(\omega,t)=\kappa(\omega)\widehat u(\omega,t)
        \end{equation*}
        其中符号函数$\kappa(\omega)=\sum_{j=0}^4 a_j(i\omega)^j$。

        问题良定的充要条件是:存在实常数$\alpha$,使得对所有实$\omega$有:
        \begin{equation*}
            \operatorname{Re}\kappa(\omega)\le\alpha
        \end{equation*}

        展开$\kappa(\omega)$的实部:
        \begin{align*}
            \kappa(\omega)&=a_0 + i a_1 \omega - a_2 \omega^2 - i a_3 \omega^3 + a_4 \omega^4 \\
            \operatorname{Re}\kappa(\omega)&=\operatorname{Re}a_0 - \operatorname{Re}a_2\omega^2 + \operatorname{Re}a_4\omega^4
        \end{align*}

        当$\operatorname{Re}a_4<0$时,$\operatorname{Re}\kappa(\omega)\sim \operatorname{Re}a_4\omega^4\to -\infty$(当$|\omega|\to\infty$),
        因此$\operatorname{Re}\kappa(\omega)$在$\mathbb{R}$上有上界,问题总是良定的。

    \section*{补充作业1}
        在时空单元$[x_{j-1/2},x_{j+1/2}]\times[t_n,t_{n+1}]$上积分PDE $u_t = u_{xx} + f(x,t)$:
        \begin{equation*}
            \iint u_t dxdt = \iint u_{xx} dxdt + \iint f(x,t) dxdt
        \end{equation*}

        左端:
        \begin{equation*}
            \iint u_t dxdt = \int_{x_{j-1/2}}^{x_{j+1/2}} [u(x,t_{n+1})-u(x,t_n)] dx
            \approx h(u_j^{n+1}-u_j^n)
        \end{equation*}

        右端第一项:
        \begin{equation*}
            \iint u_{xx} dxdt = \int_{t_n}^{t_{n+1}} [u_x(x_{j+1/2},t)-u_x(x_{j-1/2},t)] dt
        \end{equation*}
        用$\theta$加权近似时间积分:
        \begin{equation*}
            \approx \tau[\theta(u_x(x_{j+1/2},t_{n+1})-u_x(x_{j-1/2},t_{n+1})) + (1-\theta)(u_x(x_{j+1/2},t_n)-u_x(x_{j-1/2},t_n))]
        \end{equation*}
        用中心差分近似空间导数:
        \begin{equation*}
            u_x(x_{j+1/2},t)\approx \frac{u_{j+1}(t)-u_j(t)}{h}
        \end{equation*}
        于是:
        \begin{equation*}
            \iint u_{xx} dxdt \approx \frac{\tau}{h}[\theta(\delta_x^2 u_j^{n+1}) + (1-\theta)(\delta_x^2 u_j^n)]
        \end{equation*}
        其中$\delta_x^2 u_j = u_{j+1}-2u_j+u_{j-1}$。

        右端第二项:
        \begin{equation*}
            \iint f(x,t) dxdt \approx h\tau[\theta f_j^{n+1} + (1-\theta)f_j^n]
        \end{equation*}

        整理得$\theta$-格式:
        \begin{equation*}
            \frac{u_j^{n+1}-u_j^n}{\tau} = \frac{1}{h^2}[\theta\delta_x^2 u_j^{n+1} + (1-\theta)\delta_x^2 u_j^n] + \theta f_j^{n+1} + (1-\theta)f_j^n
        \end{equation*}

        截断误差分析:
        空间中心差分误差:$\delta_x^2 u_j = h^2 u_{xx} + \frac{h^4}{12}u_{xxxx} + O(h^6)$
        时间差商:$\frac{u_j^{n+1}-u_j^n}{\tau} = u_t + \frac{\tau}{2}u_{tt} + O(\tau^2)$
        当$\theta=\frac{1}{2}$时,时间截断误差为$O(\tau^2)$,空间截断误差为$O(h^2)$。

    \section*{补充作业2}
        (A) 在时空单元$[t_{n-1},t_{n+1}]\times[x_{j-1/2},x_{j+1/2}]$上积分PDE $u_t = u_{xx}$:
        \begin{equation*}
            \iint u_t dxdt = \iint u_{xx} dxdt
        \end{equation*}

        左端:
        \begin{equation*}
            \iint u_t dxdt = \int_{x_{j-1/2}}^{x_{j+1/2}} [u(x,t_{n+1})-u(x,t_{n-1})] dx
            = h(\overline u_j^{n+1}-\overline u_j^{n-1})
        \end{equation*}
        其中$\overline u_j^n = \frac{1}{h}\int_{x_{j-1/2}}^{x_{j+1/2}} u(x,t_n) dx$为单元平均。

        右端:
        \begin{equation*}
            \iint u_{xx} dxdt = \int_{t_{n-1}}^{t_{n+1}} [u_x(x_{j+1/2},t)-u_x(x_{j-1/2},t)] dt
        \end{equation*}
        用时间中点近似:
        \begin{equation*}
            \approx 2\tau[u_x(x_{j+1/2},t_n)-u_x(x_{j-1/2},t_n)]
        \end{equation*}
        用中心差分近似空间导数:
        \begin{equation*}
            u_x(x_{j+1/2},t_n)\approx \frac{\overline u_{j+1}^n - \overline u_j^n}{h}
        \end{equation*}
        于是:
        \begin{equation*}
            \iint u_{xx} dxdt \approx \frac{2\tau}{h}(\overline u_{j+1}^n - 2\overline u_j^n + \overline u_{j-1}^n)
        \end{equation*}

        得到格式:
        \begin{equation*}
            \frac{\overline u_j^{n+1} - \overline u_j^{n-1}}{2\tau} = \frac{\overline u_{j+1}^n - 2\overline u_j^n + \overline u_{j-1}^n}{h^2}
        \end{equation*}

        截断误差分析:
        时间方向泰勒展开:
        \begin{align*}
            u(x_j,t_{n+1}) &= u + \tau u_t + \frac{\tau^2}{2}u_{tt} + \frac{\tau^3}{6}u_{ttt} + O(\tau^4) \\
            u(x_j,t_{n-1}) &= u - \tau u_t + \frac{\tau^2}{2}u_{tt} - \frac{\tau^3}{6}u_{ttt} + O(\tau^4)
        \end{align*}
        于是:
        \begin{equation*}
            \frac{u(x_j,t_{n+1})-u(x_j,t_{n-1})}{2\tau} = u_t + \frac{\tau^2}{6}u_{ttt} + O(\tau^4)
        \end{equation*}
        空间方向:
        \begin{equation*}
            \frac{u(x_{j+1},t_n)-2u(x_j,t_n)+u(x_{j-1},t_n)}{h^2} = u_{xx} + \frac{h^2}{12}u_{xxxx} + O(h^4)
        \end{equation*}
        由PDE $u_t=u_{xx}$,局部截断误差为:
        \begin{equation*}
            \text{LTE} = \frac{\tau^2}{6}u_{ttt} - \frac{h^2}{12}u_{xxxx} + O(\tau^4 + h^4)
        \end{equation*}
        故格式精度为$O(\tau^2 + h^2)$。

        (B) 考虑三点线性组合$L_h[u]_j = \alpha u_{j-1} + \beta u_j + \gamma u_{j+1}$逼近$u_{xx}(x_j)$。
        泰勒展开:
        \begin{align*}
            u_{j\pm1} &= u_j \pm h u_x + \frac{h^2}{2}u_{xx} \pm \frac{h^3}{6}u_{xxx} + \frac{h^4}{24}u_{xxxx} + O(h^5) \\
            L_h[u]_j &= (\alpha+\beta+\gamma)u_j + (\gamma-\alpha)h u_x + \frac{\alpha+\gamma}{2}h^2 u_{xx} \\
            &\quad + \frac{\gamma-\alpha}{6}h^3 u_{xxx} + \frac{\alpha+\gamma}{24}h^4 u_{xxxx} + O(h^5)
        \end{align*}
        要逼近$u_{xx}$,需满足:
        \begin{align*}
            \alpha+\beta+\gamma &= 0 \\
            \gamma-\alpha &= 0 \\
            \frac{\alpha+\gamma}{2}h^2 &= 1
        \end{align*}
        解得$\alpha=\gamma=\frac{1}{h^2}$,$\beta=-\frac{2}{h^2}$。
        此时$u_{xxx}$项系数为$0$,但$u_{xxxx}$项系数为$\frac{h^2}{12}$,故误差主项为$O(h^2)$,无法达到三阶精度。

    \section*{补充作业3}
        方程$u_t = \partial_x(\kappa(x) u_x)$,其中$\kappa(x)=0.1+\sin^2 x$。

        定义空间节点$x_j=jh$,时间节点$t_n=n\tau$,界面点$x_{j+1/2}=(j+1/2)h$。
        界面扩散系数取精确值:
        \begin{equation*}
            \kappa_{j+1/2} = \kappa(x_{j+1/2}) = 0.1 + \sin^2(x_{j+1/2})
        \end{equation*}
        数值通量:
        \begin{equation*}
            F_{j+1/2}^n = \kappa_{j+1/2}\frac{u_{j+1}^n - u_j^n}{h}
        \end{equation*}
        空间离散算子:
        \begin{equation*}
            \mathcal{L}_h(u^n)_j = \frac{F_{j+1/2}^n - F_{j-1/2}^n}{h}
            = \frac{1}{h}\left[\kappa_{j+1/2}\frac{u_{j+1}^n-u_j^n}{h} - \kappa_{j-1/2}\frac{u_j^n-u_{j-1}^n}{h}\right]
        \end{equation*}
        Crank-Nicolson时间离散:
        \begin{equation*}
            \frac{u_j^{n+1}-u_j^n}{\tau} = \frac{1}{2}[\mathcal{L}_h(u^{n+1})_j + \mathcal{L}_h(u^n)_j]
        \end{equation*}

        精度分析:
        空间离散误差:在界面$x_{j+1/2}$处,
        \begin{equation*}
            \kappa_{j+1/2}\frac{u_{j+1}-u_j}{h} = \kappa(x_{j+1/2})[u_x(x_{j+1/2}) + O(h^2)]
        \end{equation*}
        故通量近似误差为$O(h^2)$,空间离散整体误差为$O(h^2)$。
        时间离散采用Crank-Nicolson格式,截断误差为$O(\tau^2)$。
        综上,格式精度为$O(\tau^2 + h^2)$。

\end{document}