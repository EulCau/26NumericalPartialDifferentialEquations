\documentclass[12pt]{article}
\usepackage{ctex}
\usepackage{amsmath,amssymb,amsthm}
\usepackage{geometry}
\geometry{a4paper,margin=1in}
\usepackage{enumitem}
\usepackage{titlesec}

\begin{document}
    \noindent{\LARGE\textbf{附录}}

    \begin{enumerate}
        \item   \textbf{Theorem} 1.3.1 (2D)
        
                The interpolation problem has the unique solution
                \begin{equation*}
                    \tilde{u}(\omega, \eta) = \frac{1}{2\pi} (e^{i\omega x} e^{i\eta y}, u)_h, \quad |\omega| \leq N/2, \, |\eta| \leq M/2.
                \end{equation*}

                \textbf{Proof:}

                The interpolant is given by
                \begin{equation*}
                    \text{Int}_{N,M} u(x,y) = \frac{1}{2\pi} \sum_{\omega=-N/2}^{N/2} \sum_{\eta=-M/2}^{M/2} \tilde{u}(\omega, \eta) e^{i\omega x} e^{i\eta y}.
                \end{equation*}
                At grid points $(x_j, y_k)$, we have:
                \begin{equation*}
                    \text{Int}_{N,M} u(x_j, y_k) = \frac{1}{2\pi} \sum_{\omega,\eta} \tilde{u}(\omega, \eta) e^{i\omega x_j} e^{i\eta y_k}.
                \end{equation*}
                Substituting $\tilde{u}(\omega, \eta) = \frac{1}{2\pi} h_x h_y \sum_{j',k'} e^{i\omega x_{j'}} e^{i\eta y_{k'}} u_{j',k'}$, we get:
                \begin{equation*}
                    \text{Int}_{N,M} u(x_j, y_k) = \frac{1}{4\pi^2} h_x h_y \sum_{j',k'} u_{j',k'} \sum_{\omega} e^{i\omega (x_j - x_{j'})} \sum_{\eta} e^{i\eta (y_k - y_{k'})}.
                \end{equation*}
                Using the orthogonality relations:
                \begin{equation*}
                    \sum_{\omega=-N/2}^{N/2} e^{i\omega (x_j - x_{j'})} = N \delta_{jj'}, \quad \sum_{\eta=-M/2}^{M/2} e^{i\eta (y_k - y_{k'})} = M \delta_{kk'},
                \end{equation*}
                and noting that $h_x h_y N M = (2\pi)(2\pi) = 4\pi^2$, we obtain:
                \begin{equation*}
                    \text{Int}_{N,M} u(x_j, y_k) = u_{j,k}.
                \end{equation*}
                Thus, the interpolant matches the grid function at all points. Uniqueness follows from the linear independence of the Fourier basis functions.
                \newpage

        \item   \textbf{Theorem} 1.3.2 (2D)

                Let
                \begin{equation*}
                    \text{Int}_{N,M} u^{(j)} = \frac{1}{2\pi} \sum_{\omega=-N/2}^{N/2} \sum_{\eta=-M/2}^{M/2} \tilde{u}^{(j)}(\omega, \eta) e^{i\omega x} e^{i\eta y}, \quad j = 1, 2,
                \end{equation*}
                interpolate the two grid functions. Then
                \begin{equation*}
                    (u^{(1)}, u^{(2)})_h = \sum_{\omega=-N/2}^{N/2} \sum_{\eta=-M/2}^{M/2} \tilde{u}^{(1)}(\omega, \eta) \overline{\tilde{u}^{(2)}(\omega, \eta)} = (\text{Int}_{N,M} u^{(1)}, \text{Int}_{N,M} u^{(2)}),
                \end{equation*}
                where the left-hand side is the discrete inner product and the right-hand side is the continuous $L^2$ inner product over $[0, 2\pi] \times [0, 2\pi]$.

                \textbf{Proof:}
  
                By definition of the discrete inner product and Fourier coefficients:
                \begin{equation*}
                    (u^{(1)}, u^{(2)})_h = h_x h_y \sum_{j,k} u^{(1)}_{j,k} u^{(2)}_{j,k}.
                \end{equation*}
                From the discrete Parseval identity for 2D DFT, we have:
                \begin{equation*}
                    (u^{(1)}, u^{(2)})_h = \sum_{\omega,\eta} \tilde{u}^{(1)}(\omega, \eta) \overline{\tilde{u}^{(2)}(\omega, \eta)}.
                \end{equation*}
                Now, consider the continuous inner product:
                \begin{equation*}
                    (\text{Int}_{N,M} u^{(1)}, \text{Int}_{N,M} u^{(2)}) = \int_0^{2\pi} \int_0^{2\pi} \text{Int}_{N,M} u^{(1)} \overline{\text{Int}_{N,M} u^{(2)}} \, \text{d}x \text{d}y.
                \end{equation*}
                Substituting the interpolant expressions:
                \begin{equation*}
                    \text{Int}_{N,M} u^{(1)} = \frac{1}{2\pi} \sum_{\omega,\eta} \tilde{u}^{(1)}(\omega, \eta) e^{i\omega x} e^{i\eta y}, \quad
                    \text{Int}_{N,M} u^{(2)} = \frac{1}{2\pi} \sum_{\omega',\eta'} \tilde{u}^{(2)}(\omega', \eta') e^{i\omega' x} e^{i\eta' y},
                \end{equation*}
                we get:
                \begin{equation*}
                    \text{Int}_{N,M} (u^{(1)}, u^{(2)}) = \frac{1}{4\pi^2} \sum_{\omega,\eta} \sum_{\omega',\eta'} \tilde{u}^{(1)}(\omega, \eta) \overline{\tilde{u}^{(2)}(\omega', \eta')} \int_0^{2\pi} e^{i(\omega - \omega')x} \text{d}x \int_0^{2\pi} e^{i(\eta - \eta')y} \text{d}y.
                \end{equation*}
                Using the orthogonality:
                \begin{equation*}
                    \int_0^{2\pi} e^{i(\omega - \omega')x} \text{d}x = 2\pi \delta_{\omega \omega'}, \quad \int_0^{2\pi} e^{i(\eta - \eta')y} \text{d}y = 2\pi \delta_{\eta \eta'},
                \end{equation*}
                we obtain:
                \begin{equation*}
                    (\text{Int}_{N,M} u^{(1)}, \text{Int}_{N,M} u^{(2)}) = \sum_{\omega,\eta} \tilde{u}^{(1)}(\omega, \eta) \overline{\tilde{u}^{(2)}(\omega, \eta)}.
                \end{equation*}
                Thus, the equality holds.
                \newpage

        \item   \textbf{Theorem} 1.3.3 (2D)

                Let $\text{Int}_{N,M} u$ be the interpolant of a grid function $u$. Then
                \begin{equation*}
                    \|\text{Int}_{N,M} u\|^2 = \sum_{\omega=-N/2}^{N/2} \sum_{\eta=-M/2}^{M/2} |\tilde{u}(\omega, \eta)|^2 = \|u\|_h^2, \tag{1.3.4}
                \end{equation*}
                and for $l = 0, 1, \ldots$,
                \begin{equation*}
                    \|D_{+x}^l u\|_h^2 \leq \left\| \frac{\partial^l}{\partial x^l} \text{Int}_{N,M} u \right\|^2 \leq \left( \frac{\pi}{2} \right)^{2l} \|D_{+x}^l u\|_h^2,
                \end{equation*}
                \begin{equation*}
                    \|D_{+y}^l u\|_h^2 \leq \left\| \frac{\partial^l}{\partial y^l} \text{Int}_{N,M} u \right\|^2 \leq \left( \frac{\pi}{2} \right)^{2l} \|D_{+y}^l u\|_h^2,
                \end{equation*}
                where $D_{+x}$ and $D_{+y}$ are the forward difference operators in the $x$- and $y$-directions, respectively.

                \textbf{Proof:}  
                The first equality (1.3.4) follows directly from the Parseval identity and the definition of the norms:
                \begin{equation*}
                    \|\text{Int}_{N,M} u\|^2 = (\text{Int}_{N,M} u, \text{Int}_{N,M} u) = \sum_{\omega,\eta} |\tilde{u}(\omega, \eta)|^2,
                \end{equation*}
                and
                \begin{equation*}
                    \|u\|_h^2 = (u, u)_h = \sum_{\omega,\eta} |\tilde{u}(\omega, \eta)|^2.
                \end{equation*}

                For the derivative estimates, we consider the $x$-direction; the $y$-direction is similar. The partial derivative of the interpolant is:
                \begin{equation*}
                    \frac{\partial^l}{\partial x^l} \text{Int}_{N,M} u(x,y) = \frac{1}{2\pi} \sum_{\omega,\eta} (i\omega)^l \tilde{u}(\omega, \eta) e^{i\omega x} e^{i\eta y}.
                \end{equation*}
                Thus, the $L^2$ norm is:
                \begin{equation*}
                    \left\| \frac{\partial^l}{\partial x^l} \text{Int}_{N,M} u \right\|^2 = \int_{[0,2\pi]^2}\left| \frac{\partial^l}{\partial x^l} \text{Int}_{N,M} u \right|^2 = \sum_{\omega,\eta} |i\omega|^{2l} |\tilde{u}(\omega, \eta)|^2 = \sum_{\omega,\eta} \omega^{2l} |\tilde{u}(\omega, \eta)|^2.
                \end{equation*}

                The forward difference operator $D_{+x}$ is defined as:
                \begin{equation*}
                    D_{+x} u_{j,k} = \frac{u_{j+1,k} - u_{j,k}}{h_x},
                \end{equation*}
                and similarly for higher orders. The discrete Fourier transform of $D_{+x}^l u$ is:
                \begin{equation*}
                    \widetilde{D_{+x}^l u}(\omega, \eta) = \left( \frac{e^{i\omega h_x} - 1}{h_x} \right)^l \tilde{u}(\omega, \eta).
                \end{equation*}
                Therefore,
                \begin{equation*}
                    \|D_{+x}^l u\|_h^2 = \sum_{\omega,\eta} \left| \frac{e^{i\omega h_x} - 1}{h_x} \right|^{2l} |\tilde{u}(\omega, \eta)|^2.
                \end{equation*}

                Now, note that for $|\omega| \leq N/2$, we have $|\omega h_x| \leq \pi$. Using the inequalities:
                \begin{equation*}
                    \left| \frac{e^{i\omega h_x} - 1}{h_x} \right| = \frac{2|\sin(\omega h_x / 2)|}{h_x} \leq |\omega|,
                \end{equation*}
                and
                \begin{equation*}
                    |\omega| \leq \frac{\pi}{2} \left| \frac{e^{i\omega h_x} - 1}{h_x} \right|,
                \end{equation*}
                which follows from $|\sin(\theta)| \geq \frac{2|\theta|}{\pi}$ for $|\theta| \leq \pi/2$ (with $\theta = \omega h_x / 2$).

                Thus,
                \begin{equation*}
                    \|D_{+x}^l u\|_h^2 = \sum_{\omega,\eta} \left| \frac{e^{i\omega h_x} - 1}{h_x} \right|^{2l} |\tilde{u}|^2 \leq \sum_{\omega,\eta} |\omega|^{2l} |\tilde{u}|^2 = \left\| \frac{\partial^l}{\partial x^l} \text{Int}_{N,M} u \right\|^2,
                \end{equation*}
                and
                \begin{equation*}
                    \left\| \frac{\partial^l}{\partial x^l} \text{Int}_{N,M} u \right\|^2 = \sum_{\omega,\eta} |\omega|^{2l} |\tilde{u}|^2 \leq \left( \frac{\pi}{2} \right)^{2l} \sum_{\omega,\eta} \left| \frac{e^{i\omega h_x} - 1}{h_x} \right|^{2l} |\tilde{u}|^2 = \left( \frac{\pi}{2} \right)^{2l} \|D_{+x}^l u\|_h^2.
                \end{equation*}
                The same reasoning applies to the $y$-direction.
    \end{enumerate}
\end{document}
